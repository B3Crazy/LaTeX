\section{Introduction}
Video games are a central leisure activity for many teenagers. Modern titles frequently include
violence and death, often very close and in a rewarding system. This study will specifically focus on how violence and death in video games affect adolescents' 
behavioral and emotional responses.
\\
Although video games containing violence and mortality are widely played by adolescents, there is ongoing debate about whether these lead to changes in teenagers. 
This study aims to clarify these relationships by examining how frequency and intensity of exposure to in-game violence and death affect teenagers.


\subsection{What type of research is the objective?}

\begin{enumerate}

    \item To describe patterns of exposure to violent and death-related content in video games among teenagers.
    \item To examine the relationship between exposure to violent/death content and aggressive behaviour and anger.
    \item To investigate whether frequent exposure to violent/death content is associated with changes in empathy and moral reasoning.
    \item To explore moderating factors (e.g., gender, parental mediation, prior aggression, and game context such as competitive vs.\ narrative-driven violence).
\end{enumerate}
The following questions will be answered.
\begin{enumerate}
    \item What are the typical frequency and intensity levels of exposure to violent and death-related content in video games among teenagers in the sample?
    \item Is greater exposure to violent/death content associated with higher levels of aggression and anger?
    \item Is greater exposure associated with lower scores on measures of empathy or altered moral judgments concerning harm?
    \item Do factors such as parental mediation, type of game, or baseline temperament moderate these associations?
\end{enumerate}



\subsection{What are the important elements/ objects?}

\begin{itemize}
    \item \textbf{Violent content:} Interactive game events that depict or require aggressive acts intended to harm characters (physical fighting, shooting, severe injury), coded by intensity and realism.
    \item \textbf{Depictions of death:} In-game events where characters die, are killed, or dying is depicted, including contextual cues (graphic vs.\ non-graphic; narrative-driven vs.\ incidental).
    \item \textbf{Exposure:} Self-reported weekly hours spent playing games containing violent/death content, supplemented by a brief content checklist to gauge intensity and realism.
\end{itemize}


\subsection{How do we evaluate our research?}

In order to evaluate this research, it is important to establish a clear set of criteria, 
which measure how violence in video games influences adolescents' perception of death and violence in real life.
Firstly the study will look into whether there are measurable changes in attitude and tolerance towards violence after exposure to violent
video games. For example, if adolescents' that grew up with violent video games show reduced emotional sensitivity to violent content
or higher acceptance towards acts of aggression in everyday situations, this can be taken as an indicator of desensitization. \\
Furthermore the study will focus on investigation patterns of behavioral changes. This might include identifying if exposure to aggressive 
game content might increase aggressive and violent thoughts, language and actions in comparison to non-violent video games. Surveys, 
psychological questionnaires, as well as controlled experiments from existing literature will serve as the main tools for gathering useful information. \\
Thirdly, criteria such as the ability of young individuals to distinguish between fiction and reality will be examined. This includes assessing
the grafical depiction of violence in videogames and the realism of graphics and violent scenarios, aswell as reactions of NPCs to player driven violence.
This also includes if the normalization of violence makes it appear as an acceptable problem-solving strategy. If adolescents struggle to
differentiate between in-game violence and real-world consequences, this could indicate a blurring of boundaries. \\
Finally the study will evaluate against existing standards of psychological and sociological analysis. The reliability of finding consistency across
existing studies and the amount to which the findings can be generalized to a broader population will be considered as essential criteria for the
validity of the research.

\subsection{Which (types of) sources are to be used for the research?}

This research will rely on various types of academic and empirical sources in order to provide a well-rounded and evidence-based perspective on the topic.
Peer-reviewed journal articles and empirical studies in the field of psychology, media studies, as well as sociology will serve as the primary sources.
These sources will provide scientific evidential data and insights into the psychological impact of violent video games on adolescents. \\
Secondly, official reports and publications from reputable organisations, such as for example the American Psychological Association (APA)
and the World Health Organization (WHO)  will be consulted, as they provide standardized guidelines as well as comprehensive and professional assessments,
regarding youth, media and mental health. \\
Furthermore, books, as well as theoretical works by experts on media effects, aggression and adolescent development will be consulted to provide a deeper
understanding of the underlying theories, concepts and frameworks that inform the research. \\
Additionally, surveys, case studies and pilot studies that are conducted by universities or youth organizations will be used to offer a valuable insight
into current trends and real-life observations. \\
Also taken into account will be Laws and age restrictions (eg. PEGI ESRB ratings, etc.) as well as parental and ethical guidelines regarding video games, 
since they show how society attempts to regulate exposure to violent content among adolescents. \\
Finally, statistical data from governmental and educational institutions will be used to provide support to the claims about the prevalence of gaming amongst
adolescents and potential correlations with violent behavior. Through combining theoretical, statistical, empirical, and practical sources, the research aims
to build a comprehensive understanding of the influence of violent video games on adolescents' perception of death and violence in real life.