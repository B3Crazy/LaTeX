\documentclass[]{report}
\renewcommand\thesection{\arabic{section}}%for page numbering in arabics
\usepackage{graphicx,tabularx,wrapfig}%for figures and tables
\usepackage[utf8]{inputenc} %allows special characters such as ä, ö, ỳ
\usepackage[english]{babel}  %set the language to English
\usepackage[margin=1.5in]{geometry} %change page margins 
\usepackage{sectsty}%section headers
\allsectionsfont{\sffamily\large}
\subsectionfont{\sffamily\normalsize}
\linespread{1.2}% line distance
\usepackage{lipsum}% http://ctan.org/pkg/lipsum
\usepackage{caption}%use for captions on tables
%use this exact command. The style and bibliographystyle has to be authoryear (Havard). The sorting is nyt: name, year, title so that the bibliography is sorted alphabetically. firstinits=true shortens the names: Albert Einstein -> A. Einstein
\usepackage[backend=bibtex,style=authoryear,bibstyle=authoryear,sorting=nyt,firstinits=true]{biblatex}
\setlength\parindent{0pt}%include this so that your paragraphs don't indent automatically
\addbibresource{referances.bib} %this attaches your bib-file, your bibliography (must be in the same folder)
\usepackage[compact]{titlesec}%include title formatting package

% Title Page
\title{How does violence and death in video games affect adolescents' perception of death and violence in real life in the area of North Rhine-Westphalia?}
\author{Paul Szelag (5496535) and Anna Varga (5378192)}
\date{December 9th 2025 \\Module: ARDA \\Fontys University of Applied Science - Informatics \\Venlo, Limburg, Netherlands}


\begin{document}
\maketitle

\begin{abstract}
Violent and death related content in video games raises concerns about potential impacts on adolescents' moral reasoning, emotional responses and behavioural tendencies.
This study investigated whether exposure to violent video game content influences these developmental domains in North Thine-Westphalia, Germany. A total of 118
adolescents (aged 12-20) and an approximately equal number of parents completed structured questionnaires assessing gaming habits, engagement patterns, genre preferences,
emotional responses to in-game violence and behavioural aggression tendencies.

Results revealed moderate gaming engagement (57\% playing 0.5-3 hours daily) with substantial exposure to violent content, particularly shooters (61\% engagement).
Gender differentiation was pronounced with males reporting significantly higher enjoyment of violent gameplay (85\% vs. 76\% combined; 62\% vs. 42\% consistent enjoyment).
Gaming demonstrated predominantly social characteristics (67\% with substantial peer interaction). Regarding behavioural outcomes, 70\% of parents reported no aggression
increase or actual decrease following gaming, while 30\% observed increases (23\% moderate, 7\% significant).

Evaluation of three research hypotheses derived from desensitization and aggression frameworks yielded limited support for Hypothesis 1 (violent game exposure correlates
with higher aggression/ reduced empathy), insufficient measurement for Hypothesis 2 (reward system amplification) and robust support for differential exposure, but
limited outcome differentiation in Hypothesis 3 (gender effects). THe dissociation between high in-game violence enjoyment (76\%) and predominantly unchanged real-world
aggression suggests adolescents maintain psychological boundaries between fictional gameplay and real-world moral domains - a phenomenon termed \textit{compartmentalization}
(\cite{Békés_Ferstenberg_Perry_2018}).

However the 30\% experiencing aggression increases, particularly the 7\% with significant escalation, indicates that violent content demonstrably affects a substantial
minority. These heterogenous outcomes challenge linear dose-response media effects models and suggest that protective factors (social gaming, moral development, individual
differences) substantially buffer against predicted negative effects for most adolescents. THe findings necessitate reconceptualization of media effects frameworks
emphasizing domain-specific emotional calibration and individual heterogeneity rather than uniform desensitization trajectories. Practical implementations emphasize
informed parental engagement, media literacy education and ethically sophisticated game design over categorical content prohibition.
\pagenumbering{roman}
\end{abstract}

\tableofcontents
\setcounter{page}{3}
%\listoffigures %UNCOMMENT IF YOU HAVE FIGURES
%\listoftables %UNCOMMENT IF YOU HAVE TABLES
\pagebreak
\pagenumbering{arabic}	
	
\section{Introduction}
Video games have become one of the most prevalent forms of entertainment among adolescents across the globe,
with violent and death-related content present in many popular titles (\textit{e.g. Call of Duty, Fortnite, Grand Theft Auto}).
 Existing rating systems such as PEGI (EU) and ESRB (US) already flag the degree of violent realism in these games,
 reflecting widespread public concern about the effects of such content on young people's development and moral reasoning.

 \subsection{Background and significance}
 Debates about whether violent video games contribute to aggressive attitudes or behavior have persisted for as long as such games exist.
 Meta-analyses and longitudinal studies have produced mixed results, with some finding small to none effects on aggression, whilst others
 suggest that frequent exposure to violent or dark-themed games may increase tolerance toward aggression and reduce empathic concern for
 others \cite{anderson2007violent}. 
\section{Methodology}
\subsection{Research Design}
This study will employ a mixed-methods approach, combining a quantitative, cross-sectional correlational design with exploratory moderation analyses and examination
of potential relevant existing datasets. The primary data collection will involve administering standardized questionnaires with the aim to assess whether adolescents'
exposure to violent and death-related content can be associated with changing perceptions of violence, moral reasoning, emotional responses and behavioral tendencies.

\subsection{Participants}
The target for this study is adolescents aged 12 to 20 years. A sample size of approximately 100-200 participants will be recruited from secondary schools and clubs
to ensure a sufficient and diverse representation. The geo-locational focus will be the set on urban and suburban regions within Nordrhine-Westphalia, Germany. \\
For inclusion in the study, participants must access to video games at least twice a week. To ensure ethical compliance, informed consent will be obtained from all participants
and participation will be voluntary, anonymous and in accordance with ethical guidelines. For participants under the age of 14, parental consent will also be required.

\subsection{Materials and Measures}
A structured questionaire will be developed to assess the key constructs of interest. The questionnaire will include validated scales and self-report items to measure:
\begin{itemize}
    \item \textbf{Exposure to video games:} Frequency and duration of video games played, with a focus on violent and death-related content.
    \subitem e.g., "How often do you play video games?", "Do you play games that involve violence or death?"
    \item \textbf{Rewarding and supporting of aggressive acts:} Assessment of the extent to which the games played reward and support aggressive behavior.
    \subitem e.g., "In these games, how often does the gameplay require you to perform aggressive actions to succeed?", "In these games, how/ how often are aggressive
    actions rewarded?"
    \item \textbf{Moral reasoning about violence:} Using established scales such as the Aggression Acceptance Scale (AAS, \cite{Anderson_Bushman_2018}) to measure attitudes
    toward aggression.
    \subitem e.g., "Is it acceptable to hit someone who insults you?", "Is it okay to use violence to solve problems?"
    \item \textbf{Perception of death and empathy for pain:} Utilizing instruments like the Basic Empathy Scale (BES) or the Empathy for Pain Questionnaire
    (\cite{Miedzobrodzka_2023}) to assess emotional and cognitive empathy toward others' suffering.
    \subitem e.g., "When I see someone get hurt, I feel sorry for them.", "I can understand how others feel when they are in pain."
    \item \textbf{Demographics and Gender:} Collecting information on the participants age, gender and demographic background to explore potential moderating effects.
    \item \textbf{Control variables:} Including measures of average weekly playtime, game genre preference (competitive, narrative, cooperative), parental mediation
    (restrictive / active), and prior aggression incidents.
\end{itemize}

All questionnaires will be administered in English and will have a German translation available to ensure comprehension amongst all participants.

\subsection{Procedure}
The Procedure follows the same structure for all participants. Access will be made available with a link to the online questionnaire.
\begin{enumerate}
    \item \textbf{Distribution:} The distribution will be handled through multiple channels to reach a diverse and representative sample of adolescents.
        \subitem Schools: A person of authority will br contacted to distribute the link to students.
        \subitem Clubs: Trainers or leaders will be contacted to distribute the link to members.
    \item \textbf{Consent:} Prior to participation the questionnaire will inform the adolescents that participation is entirely voluntary. It will inform that children under
    the age of 14 have to take the survey with parental consent. Anonymity and confidentiality will be assured.
    \item \textbf{Questionnaire sequence:}
        \subitem Demographics, age and gender
        \subitem Video game exposure and habits
        \subitem Perception of violence, empathy and moral disengagement
    \item \textbf{Debriefing:} After finishing the questionnaires participants will receive an explanation on the purpose of the survey.
\end{enumerate}
\section{Results}

\subsection{Sample Characteristics}

\subsubsection{Age Distribution}

\begin{wrapfigure}{r}{0.45\textwidth}
    \centering
    \includegraphics[width=0.45\textwidth]{images/Ages.png}
    \caption{Age Distribution of Student and Parent Participants}
    \label{fig: Age Distribution}
\end{wrapfigure}
A total of 118 participants were included in this study: 118 students (aged 12-20 years), with corresponding data from parents. As shown in Figure 1, the student sample 
was concentrated in the younger to mid-adolecent age ranges with relatively even distribution. The 12-14 age group accounted for 35\% of students (36\% of parents), 
the 15-17 age group accounted for 40\% of students (36\% of parents), and the 18-20 age group accounted for 25\% of students (28\% of parents). This shows a good spread
accross the adolescent age range with a slight but negligible underrepresentation of older adolescents. The close alignment between student and parent age reporting demonstrates convergent validity and confirms that parent-child correspondence
was maintained in the data collection process.\\

\subsubsection{Gender Distribution}

\begin{wrapfigure}{r}{0.45\textwidth}
    \centering
    \includegraphics[width=0.45\textwidth]{images/Gender.png}
    \caption{Gender Distribution of Student Participants}
    \label{fig: Gender Distribution}
\end{wrapfigure}
As presented in Figure 2, gender distribution was relatively balanced among student participants, with males representing a slight majority. The difference of 7\% between
male (52\%) and female (45\%) students was minimal, indicating a fairly even gender representation in the sample and as such does not suggest a signifficant gender bias in
participation. An additional 3\% of students identified as non-binary or preferred not to disclose their gender. This balanced distribution enables exploration of gender
differences across gaming behaviours and outcomes.\\

\subsection{Gaming Duration and Frequency}

\subsubsection{Average Daily Gaming Time}

\begin{wrapfigure}{r}{0.45\textwidth}
    \centering
    \includegraphics[width=0.45\textwidth]{images/Hours played.png}
    \caption{Average Daily Gaming Time Among Students}
    \label{fig: Daily Gaming Time}
\end{wrapfigure}
Figure 3 represents the distribution of average daily gaming hours across the students. The data reveals a distinctly tendency toward moderat engagement patterns: 30\% 
reported 1-2 hours, 30\% reported 2-3 hours, and 20\% reported less than 1 hour of gaming per day. Only a small minority (8\%) reported gaming for more than 4 hours daily. 
This indicates that while gaming is a common activity, excessive daily gaming is relatively uncommon in this adolescent sample. Cumulatively, 75\% of students reported
engaging in gaming for 0.5-3 hours daily. Extreme gaming intensities (exceeding 5 hours daily) were rare, reported by only 3\% of participants, whilst 5\% reported
4-5 hours and 12\% reported 3-4 hours. The distribution approximated normality around a modal value of 2-4 hours daily, whith extrem outliers representing distribution tails
rather than normative engagement patterns.\\
Notably, these moderate patterns contradict assumptions about widespread gaming addiction. The finding that the modal category of 2-4 hours falls within leisure-time
recommendations and that exessive gaming (>5 hours) remains rare suggests that gaming constitutes a normative leisure activity rather than a pathological behaviour
in this adolescent cohort.\\

\subsubsection{Gaming Frequency Per Week}

\begin{wrapfigure}{r}{0.45\textwidth}
    \centering
    \includegraphics[width=0.45\textwidth]{images/How many days played in week.png}
    \caption{Gaming Frequency Per Week Among Students}
    \label{fig: Gaming Frequency}
\end{wrapfigure}
As illustrated in Figure 4, the frequency with which students engage in gaming across a typica week shows a great concentration in the mid-range reaching as high as 24\%
at 3 days a week. Specifically, 5\% of students reported gaming 0 days a week, which correlates who reported 0 hours of daily gaiming. The largest proportion of students
(48\%) reported gaming 3-5 days per week, with 24\% gaming 3 days, 19\% gaming 4 days and 15\% gaminig 5 days weekly. A further 16\% reported gaming 6 (10\%) -7 (6\%) days 
per week, indicating daily or near-daily engagement. Another 21\% reported gaming 1 (8\%) - 2 (14\%) days weekly. This distribution indicates that while daily gaming is
not the norm a substantial minority of students engage in gaming on most days of the week.\\
The heterogenous frequency distribution, combined with moderate daily gaming durations, indicates substential interindividual variation in cumulative gaming intensity.
When combining frequency and duration data, this exposure can range from minimal (e.g., 0.5 hours on 1 day) to substantial (e.g., 5+ hours on seven days). This variability
highlights the importance of considering both dimensions when assessing gaming behaviours and potential impacts.\\

\subsection{Gaming Context and Social Engagement}

\subsubsection{Solo Versus Social Engagement}

\begin{wrapfigure}{r}{0.45\textwidth}
    \centering
    \includegraphics[width=0.45\textwidth]{images/Do they play alone or with friends.png}
    \caption{Solo Versus Social Gaming Engagement Among Students}
    \label{fig: Solo vs Social Gaming}
\end{wrapfigure}
Figure 5 presents the student and parent perspectives on whether adolecents played alone or with friends. A striking difference emerges between student self-reports and parent
perceptions. From the students perspective, 12\% reported gaming solely with friends, as much as 29\% reported gaming primarily with friends, 26\% reported gaming equally alone
and with friends and 18\% reported playing primarily alone. Only 15\% reported gaming exclusively alone. In contrast, parents precieved a much higher prevalence of solitary
games with 19\% reporting their child games exclusively alone and as much as 31\% reported their child gaming primarily alone. Only 8\% said their child only plays with friends
and another 18\% said their child primarily plays with friends. This discrepancy suggests that parents may underestimate the social dimensions of their children's gaming
activity.\\
The data indicates that while a substantial portion of adolescents do engage in social gaming, a significant minority also games alone. The divergence between student and parent
reports highlights potential gaps in parental awareness of their children's gaming contexts. This has implications for understanding the social versus solitary nature of adolescent 
gaming and the need for parental education regarding gaming behaviours.\\

\subsubsection{Game Preferences and Genre Selection}

\begin{wrapfigure}{r}{0.45\textwidth}
    \centering
    \includegraphics[width=0.45\textwidth]{images/Type of games played.png}
    \caption{Game Genre Preferences Among Students}
    \label{fig: Game Genre Preferences}
\end{wrapfigure}
Figure 6 presents the types of games most frequently played by students. Shooter are the most popular genre, with 61\% of students reporting reglar engagement. This ios followed
by Sandbox games (49\%), Role-Playing Games (RPGs) (37\%), Open World Games (28\%), Sport games (23\%) and Story Games (19\%). This distribution indicates a strong preference for
action oriented gaming experiences among adolecents, with shooters and sandbox games dominating the landscape. The popularity of RPGs and open world games also suggests an interest
in immersive, narrative driven experiences. Less popular genres such as sports and story games indicate more niche interests. Overall, the genre preferences reflect a diverse range 
of gaming tastes within the adolescent cohort, with a clear leaning toward action and adventure oriented titles. It also suggests a tendency towards mostly shortlived and fast-paced
gaming experiences rather than slow-paced and strategic ones.\\

\subsection{Emotional Engagement with IN-Game Violence}

\subsubsection{Happiness during Violent In-Game Actions}

\begin{wrapfigure}{r}{0.45\textwidth}
    \centering
    \includegraphics[width=0.45\textwidth]{images/Happines when playing violent games.png}
    \caption{Happiness Levels During Violent In-Game Actions Among Students}
    \label{fig: Happiness during Violent Actions}
\end{wrapfigure}
As illustrated in Figure 7, students reported varying levels of happines when engaging in violent actions within games. A majority of 76\% reported feeling happy 
(46\% always, 30\% sometimes) when perfoming violent acts in games. In contrast only 24\% reported never enjoying violent acts in game or getting joy out of them.
This indicates that for most adolecents, engaging in violent gameplay is assosiated with positive emotional experiences. A notable difference can be oberved between
male and female adolescents. As for male students, as much as 62\% rewarted always enjoying violent in-geme acts and another 23\% reported sometimes enjoing them.
In contrast only 42\% of female adolescents reported always enjoying violent acts in game with another 34\% reporting sometimes enjoying them. This suggests that male
adolescents derive greater happines from violent gameplay compared to their female counterparts and are as such more emotionally engaged by violent in-game actions.\\

\subsection{Behavioral Outcomes Related to In-Game Violence}

\subsubsection{Aggression Level after prolonged Gaming}

\begin{wrapfigure}{r}{0.45\textwidth}
    \centering
    \includegraphics[width=0.45\textwidth]{images/Aggression growing after playing for a longer period.png}
    \caption{Self-Reported Aggression Levels After Prolonged Gaming Among Students}
    \label{fig: Aggression Levels}
\end{wrapfigure}

\section{Conclusion}

\subsection{Summary of Principal Findings}

This investigation examined whether violent and death-related content within video games influences adolescents' moral reasoning, emotional responses,
and behavioral tendencies in North Rhine-Westphalia, Germany. The study surveyed 118 students (aged 12-20) and an approximate equal number of parents,
addressing three research hypotheses derived from desensitization and aggression frameworks.

Key findings revealed: moderate gaming engagement (majority paying 0.5-3 hours daily, 48\% gaming 3-5 days weekly), with extreme gaming being rare
(3\% exceeding 5 hours); substantial violent content exposure, with shooters dominating preferences at 61\%; pronounced gender differentiation, with
males reporting more liking of violent in-game actions (85\% vs. 76\% combined; 62\% vs. 42\% always happy); predominantly social gaming contexts
(67\% reporting substantial social components); heterogenous aggression outcomes, with 70\% showing no increase or actual decrease post-gaming (18\% decrease).

\subsection{Evaluation of Research Hypotheses}

\textbf{Hypothesis 1} (violent exposure correlates with higher aggression/ reduced empathy) received \textbf{limited support}. Despite 61\% engaging with shooters,
and 76\% reporting positive affect during violent actions, 70\% demonstrated no aggression increase. The 30\% reporting increase indicates heterogenous rather than
uniform effects.

\textbf{Hypothesis 2} (reward systems amplify negative effects) could not be adequately evaluated due to measurement limitations assessing genres rather than specific
reward mechanics. Despite high shooter engagement, the absence of systematic aggression increase suggests reward exposure operating weakly or protective factors
buffer effectively.

\textbf{Hypothesis 3} (stronger effects among males) received \textbf{robust support for differential exposure, but limited support for differential outcomes}. Males
reported greater liking of violent content (85\% vs. 76\%) and enjoyment throughout (62\% vs. 42\%), yet aggregational outcomes showed no corresponding gender divergence.

\subsection{Addressing the Research question}

The central question asked: \textit{How does violence and death in video games affect adolescents' perception of death and violence in real life in the area of North Rhine-Westphalia?}
\textbf{Violent game exposure does not systematically corrupt adolescent behavioral control or moral reasoning for the majority.}. Despite substantial engagement, most adolescents
demonstrated no aggression increase or showed actual decreases.  The dissociation between in-game emotional engagement and real-world behavioural manifestations
indicates an upkeep of psychological boundaries between fictional gameplay and real world morality - also called \textit{adaptive compartmentalization} 
(\cite{Békés_Ferstenberg_Perry_2018}).

However, the 30\% experiencing aggression increases - particularly the 7\% with significant escalations - indicates violent content demonstrably affects a substantial
minority. This heterogeneity challenges simplistic narratives, requiring nuanced understanding of moderating factors determining individual vulnerability and resilience.

\subsection{Theoretical Implications}

The findings necessitate reconceptualization of media effects beyond linear dose-response models towards frameworks, emphasizing \textbf{domain-specific emotional
calibration and heterogenous effects moderated by individual differences}.

The coexistence of high in-game violence enjoyment (76\%) with predominantly unchanged aggression (70\%) suggest that adolescents develop context-specific emotional
responses distinguishing between fictional and real-world moral reasoning. This represents developmental achievement wherein metacognitive awareness enables
simultaneous gameplay enjoyment and maintenance of prosocial standards. Traditional desensitization (\cite{Wolpe_1968}) models, predicting emotional numbing, insufficiently account for 
this complexity.

The predominantly social gaming contexts (67\%) further highlight the importance of peer-mediated normative influences, suggesting that social interaction processes may counterbalance
potential negative effects of violent content exposure. The discrepancy between student reports (67\% social) and parental perceptions (50\% solitary) indicates stakeholders underestimate
social-collaborative dimensions buffering against individual-level desensitization.

The 70/30 distribution indicates fundamental individual heterogeneity, requiring investigation of moderating variables such as trait aggression, family-environment quality and emotional
regulation capabilities. The 7\% experiencing significant escalation may represent a vulnerable subpopulation requiring targeted investigation and intervention.

\subsection{Methodological Considerations}

The investigation incorporated multi-perspective design (student and parent reports), comprehensive assessment across multiple domains, filtered demographics and naturalistic ecological
validity. However, critical limitations constrain causal inference: the cross-sectional design precludes temporal precedence establishment; self-report measures may introduce social
desirability bias; absence of statistical controls for trait-level variables prevents definitive mechanistic identification; genre-level measurement does not capture within-genre
variation in reward structures; direct assessment of death perception and moral reasoning was not conducted. These limitations necessitate cautious interpretation and emphasize the need
for longitudinal designs, experimental manipulations and targeted moderator measurement.

\subsection{Practical Recommendations}

\begin{itemize}
    \item \textbf{For parents and educators:}
    \begin{itemize}
        \item Evidence does not support categorical prohibition but rather informed engagement through co-play and ethnical discussion
        \item behavioral monitoring focussing on observable changes rather than content alone is recommended
        \item media literacy education supporting critical evaluation of game content and emotional responses is advocated
        \item recognition of gaming's legitimate social functions is important
        \item individual assessment for the 30\% manifesting aggression increases before intervening and implementing restrictions is advised.
    \end{itemize}

    \item \textbf{For Game Developers:} 
    \begin{itemize}
        \item Design violence as consequence-generating rather than exclusively reward-producing to mitigate potential reinforcement of aggression. A good example for this strategy is the 
        game "Red Dead Redemption 2", where violent actions often lead to negative in-game consequences such as increased law enforcement attention and loss of honor points.(\cite{RDR_Honor_System})
        Incorporating moral choice systems that highlight the repercussions of violent actions can encourage players to reflect on their in-game behavior
        \item integrate prosocial collaboration alongside combat
        \item develop narrative sophistication humanizing adversaries
        \item provide granular parental controls enabling tailored content regulation based on individual child profiles
        \item commission empirical research examining how design features influence psychological outcomes
    \end{itemize}


    \item \textbf{For Policymakers:} 
    \begin{itemize}
        \item Strengthen rating system transparency with evidence based criteria
        \item fund longitudinal research establishing causal mechanisms
        \item invest in population-level media literacy initiatives
        \item support research identifying vulnerable subpopulations
        \item facilitate collaborative partnerships supporting ethically sophisticated design innovations
    \end{itemize}
\end{itemize}

\subsection{Final Remarks}
Rather than confirming fear of widespread moral corruption, this investigation reveals complex patterns: substantial violent content exposure coexists with predominantly unchanged  or decreased 
aggression for 70\%, whilst 30\ experience increase requiring targeted attention. This heterogeneity  challenges binary debates about inherent harm versus harmlessness, emphasizing the need for 
nuanced understanding of individual differences, social contexts and emotional calibration processes.

Contemporary adolescents demonstrate sophisticated psychological compartmentalization, maintaining psychological boundaries between fictional gameplay and real world morality. The predominantly 
social-collaborative nature of gaming (67\%) likely activates protective peer influences buffering against potential individual-level desensitization. However, the non-negligible minority
experiencing aggression increases, particularly the 7\% with significant escalations, indicates that violent content demonstrably affects a vulnerable subpopulation.

The path forward requires evidence-informed approaches supporting adolescent autonomy whilst safeguarding psychological well-being and recognizing legitimate concerns. Through informed parental
engagement, media literacy education, ethically sophisticated game design and targeted support for vulnerable subpopulations, interactive media can evolve towards experiences that are
simultaneously engaging and responsible. Understanding that 70\% maintain healthy boundaries whilst 30\% require support enables nuanced approaches replacing categorical prohibition with
informed engagement and individualized assessment - the approach most consistent with supporting adolescent moral development in contemporary digital environments in North Rhine-Westphalia and beyond.
\printbibliography[title={References}]
\end{document}
	
	
	
	
	
	
	
	
	
	
	
	
	
	
	
	
	
	
	
	
	
	
\end{document}          
