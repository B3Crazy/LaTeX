\section{Discussion}

\subsection{Interpretation of Principal Findings}

This investigation examined whether exposure to violent and death-related contend within video gaming influences adolecents' moral reasoning, emotional responsivity,
and behavioral manifestations within North Rhine-Westphalia, Germany. The empirical findings reveal a complex pattern engagement with violent content that requires 
nuanced interpretation in light of established media effects theoretical framework.\\

\subsubsection{Gaming Engagement Patterns and Intensity}

The study revealed moderate and heterogenous gaming engagement across the adolescents sample. With 75\% op participants engaging in gaming for 0.5 - 3 hours daily and
48\% gaming 3-5 days per week, the temporal parameters align with normative leisure-time allocations rather than pathological consumption patterns 
(\cite{Wang_Chan_Mak_Ho_Wong_Ho_2014}). The modal gaming duration of 2-4 hours daily falls within recommendations for healthy recreational activity and contradicts
assumptions regarding widespread gaming addiction in contemporary adolescent populations.

Notably, extreme gaming (exceeding 5 hours daily) remained rare at only 3\%, indicating that the intensive engagement feared in public discourse characterizes only
a minimal minority. This distribution suggests that contemporary adolescent gaming, whilst prevalent, does not demonstrate the addiction-level engagement sometimes
portrayed in media narratives. \\

\subsubsection{Genre Selection and Violent Content Exposure}

The genre distribution demonstrated that shooters represent the most frequent reported game type (61\%), followed by sandbox games (49\%), RPGs (37\%) and open world games 
(28\%). This finding indicates substantial exposure to violent content among adolescents, with shooter games -characterized by prominent violent and death-related mechanics-
representing the dominant genre preference. However, the high prevalence of sandbox and RPG games simultaneously suggests that gaming encompasses diverse experiences beyond
violence-centric mechanics, offering opportunities for creative expression and narrative engagement.

The fact that 61\% regularly engage with shooters represents considerable exposure to violent gameplay mechanics and reward systems, wherein combat constitutes a core gameplay
loop. This exposure magnitude aligns with the theoretical concern that reward-contingent violence may reinforce aggressive response patterns through operant conditioning
(\cite{Carnagey_Anderson_2005}). However, interpretation requires careful consideration of the behavioral and emotional outcomes observed in the sample. \\

\subsubsection{Social Gaming Context and Moderating Factors}

A critical distinction emerged between student self-reports and parent perceptions regarding gaming context. Student reports indicated that 47\% engaged primarily or
partially with friends (12\% solely social + 29\% primarily social + 26\% equally alone and with friends = a combined 67\% with substantial social components), whilst
parents reported substantially higher solitary engagement (50\% exclusively or primarily alone: 19\% exclusively alone + 31\% primarily alone). This discrepancy likely
reflects parental underestimation of social gaming dimensions.

The predominance of social gaming contexts (67\% from student perspective) represents a significant moderating variable. Research establishes that peer presence activates
normative social influence towards prosocial behaviour, potentially mitigating theoretically predicted negative effects of violent content exposure (\cite{fikkers_2016}).
The collaborative nature of contemporary multiplayer gaming suggests that social interaction processes my substantially buffer against individual-level desensitization.

\subsubsection{Emotional Engagement with In-Game violence}
The finding that 76\% of adolescents reported positive effects during violent in-game actions (46\% always happy + 30\% sometimes happy) (See Figure 7) demonstrates substantial
hedonic engagement with violent game mechanics. However, gender differentiation was pronounced 62\% of male adolescents reported consistently enjoying violent actions compared
to 42\% of female adolescents. This 20-percentage-point difference aligns with established literature documenting gender differences in game genre selection and violent 
content engagement (\cite{Hartmann_Möller_Krause_2015}).

Importantly, positive affective response to in-game violence reflects normative engagement with gameplay mechanics rather than necessarily indicating moral corruption or
real-world violent attitudes. The distinction between context-specific emotional engagement (happiness during combat gameplay) and real-world moral reasoning requires careful 
interpretive separation and constitutes a critical point for theoretical understanding. \\

\subsection{Evaluation of Research Hypotheses}

\subsubsection{Hypothesis 1: Violent Exposure and Aggression/ Empathy}

The first hypothesis proposed greater exposure to violent/ death-related content correlates with higher acceptance of aggression and reduced empathy. The findings provide
\textbf{limited and complex support} for this hypothesis.

Regarding behavioral aggression, the data present a nuanced picture. Following prolonged gaming, 70\% of parents reported either no aggression increase or actual aggression
decrease with only 30\% reporting increase (23\% moderate + 7\% significant) (See Figure 8). This distribution does not support predictions of systematic aggression escalation
proportional to violent content exposure. The 18\% observing aggression reduction suggests potential cathartic or stress-regulatory functions that warrant further investigation.

However, the 30\% experiencing aggression increase -particularly teh 7\% experiencing significant escalation- indicates that a non-negligible minority does manifest heightened
aggression following intensive gaming. This suggests that violent content exposure demonstrates heterogenous effects across the adolescent population rather than uniform
behavioural outcomes. Factors beyond game content likely determine whether aggression escalates or decreases, including individual trait aggression, family dynamics, peer
relationships, and emotional regulation capacities.