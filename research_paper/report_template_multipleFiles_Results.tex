\section{Results}

\subsection{Sample Characteristics}

\subsubsection{Age Distribution}

\begin{wrapfigure}{r}{0.45\textwidth}
    \centering
    \includegraphics[width=0.45\textwidth]{images/Ages.png}
    \caption{Age Distribution of Student and Parent Participants}
    \label{fig: Age Distribution}
\end{wrapfigure}
A total of 118 participants were included in this study: 118 students (aged 12-20 years), with corresponding data from parents. As shown in Figure 1, the student sample 
was concentrated in the younger to mid-adolecent age ranges with relatively even distribution. The 12-14 age group accounted for 35\% of students (36\% of parents), 
the 15-17 age group accounted for 40\% of students (36\% of parents), and the 18-20 age group accounted for 25\% of students (28\% of parents). This shows a good spread
accross the adolescent age range with a slight but negligible underrepresentation of older adolescents. The close alignment between student and parent age reporting demonstrates convergent validity and confirms that parent-child correspondence
was maintained in the data collection process.\\

\subsubsection{Gender Distribution}

\begin{wrapfigure}{r}{0.45\textwidth}
    \centering
    \includegraphics[width=0.45\textwidth]{images/Gender.png}
    \caption{Gender Distribution of Student Participants}
    \label{fig: Gender Distribution}
\end{wrapfigure}
As presented in Figure 2, gender distribution was relatively balanced among student participants, with males representing a slight majority. The difference of 7\% between
male (52\%) and female (45\%) students was minimal, indicating a fairly even gender representation in the sample and as such does not suggest a signifficant gender bias in
participation. An additional 3\% of students identified as non-binary or preferred not to disclose their gender. This balanced distribution enables exploration of gender
differences across gaming behaviours and outcomes.\\

\subsection{Gaming Duration and Frequency}

\subsubsection{Average Daily Gaming Time}

\begin{wrapfigure}{r}{0.45\textwidth}
    \centering
    \includegraphics[width=0.45\textwidth]{images/Hours played.png}
    \caption{Average Daily Gaming Time Among Students}
    \label{fig: Daily Gaming Time}
\end{wrapfigure}
Figure 3 represents the distribution of average daily gaming hours across the students. The data reveals a distinctly tendency toward moderat engagement patterns: 30\% 
reported 1-2 hours, 30\% reported 2-3 hours, and 20\% reported less than 1 hour of gaming per day. Only a small minority (8\%) reported gaming for more than 4 hours daily. 
This indicates that while gaming is a common activity, excessive daily gaming is relatively uncommon in this adolescent sample. Cumulatively, 75\% of students reported
engaging in gaming for 0.5-3 hours daily. Extreme gaming intensities (exceeding 5 hours daily) were rare, reported by only 3\% of participants, whilst 5\% reported
4-5 hours and 12\% reported 3-4 hours. The distribution approximated normality around a modal value of 2-4 hours daily, whith extrem outliers representing distribution tails
rather than normative engagement patterns.\\
Notably, these moderate patterns contradict assumptions about widespread gaming addiction. The finding that the modal category of 2-4 hours falls within leisure-time
recommendations and that exessive gaming (>5 hours) remains rare suggests that gaming constitutes a normative leisure activity rather than a pathological behaviour
in this adolescent cohort.\\

\subsubsection{Gaming Frequency Per Week}

\begin{wrapfigure}{r}{0.45\textwidth}
    \centering
    \includegraphics[width=0.45\textwidth]{images/How many days played in week.png}
    \caption{Gaming Frequency Per Week Among Students}
    \label{fig: Gaming Frequency}
\end{wrapfigure}
As illustrated in Figure 4, the frequency with which students engage in gaming across a typica week shows a great concentration in the mid-range reaching as high as 24\%
at 3 days a week. Specifically, 5\% of students reported gaming 0 days a week, which correlates who reported 0 hours of daily gaiming. The largest proportion of students
(48\%) reported gaming 3-5 days per week, with 24\% gaming 3 days, 19\% gaming 4 days and 15\% gaminig 5 days weekly. A further 16\% reported gaming 6 (10\%) -7 (6\%) days 
per week, indicating daily or near-daily engagement. Another 21\% reported gaming 1 (8\%) - 2 (14\%) days weekly. This distribution indicates that while daily gaming is
not the norm a substantial minority of students engage in gaming on most days of the week.\\
The heterogenous frequency distribution, combined with moderate daily gaming durations, indicates substential interindividual variation in cumulative gaming intensity.
When combining frequency and duration data, this exposure can range from minimal (e.g., 0.5 hours on 1 day) to substantial (e.g., 5+ hours on seven days). This variability
highlights the importance of considering both dimensions when assessing gaming behaviours and potential impacts.\\

\subsection{Gaming Context and Social Engagement}

\subsubsection{Solo Versus Social Engagement}

\begin{wrapfigure}{r}{0.45\textwidth}
    \centering
    \includegraphics[width=0.45\textwidth]{images/Do they play alone or with friends.png}
    \caption{Solo Versus Social Gaming Engagement Among Students}
    \label{fig: Solo vs Social Gaming}
\end{wrapfigure}
Figure 5 presents the student and parent perspectives on whether adolecents played alone or with friends. A striking difference emerges between student self-reports and parent
perceptions. From the students perspective, 12\% reported gaming solely with friends, as much as 29\% reported gaming primarily with friends, 26\% reported gaming equally alone
and with friends and 18\% reported playing primarily alone. Only 15\% reported gaming exclusively alone. In contrast, parents precieved a much higher prevalence of solitary
games with 19\% reporting their child games exclusively alone and as much as 31\% reported their child gaming primarily alone. Only 8\% said their child only plays with friends
and another 18\% said their child primarily plays with friends. This discrepancy suggests that parents may underestimate the social dimensions of their children's gaming
activity.\\
The data indicates that while a substantial portion of adolescents do engage in social gaming, a significant minority also games alone. The divergence between student and parent
reports highlights potential gaps in parental awareness of their children's gaming contexts. This has implications for understanding the social versus solitary nature of adolescent 
gaming and the need for parental education regarding gaming behaviours.\\

\subsubsection{Game Preferences and Genre Selection}

\begin{wrapfigure}{r}{0.45\textwidth}
    \centering
    \includegraphics[width=0.45\textwidth]{images/Type of games played.png}
    \caption{Game Genre Preferences Among Students}
    \label{fig: Game Genre Preferences}
\end{wrapfigure}
Figure 6 presents the types of games most frequently played by students. Shooter are the most popular genre, with 61\% of students reporting reglar engagement. This ios followed
by Sandbox games (49\%), Role-Playing Games (RPGs) (37\%), Open World Games (28\%), Sport games (23\%) and Story Games (19\%). This distribution indicates a strong preference for
action oriented gaming experiences among adolecents, with shooters and sandbox games dominating the landscape. The popularity of RPGs and open world games also suggests an interest
in immersive, narrative driven experiences. Less popular genres such as sports and story games indicate more niche interests. Overall, the genre preferences reflect a diverse range 
of gaming tastes within the adolescent cohort, with a clear leaning toward action and adventure oriented titles. It also suggests a tendency towards mostly shortlived and fast-paced
gaming experiences rather than slow-paced and strategic ones.\\

\subsection{Emotional Engagement with IN-Game Violence}

\subsubsection{Happiness during Violent In-Game Actions}

\begin{wrapfigure}{r}{0.45\textwidth}
    \centering
    \includegraphics[width=0.45\textwidth]{images/Happines when playing violent games.png}
    \caption{Happiness Levels During Violent In-Game Actions Among Students}
    \label{fig: Happiness during Violent Actions}
\end{wrapfigure}
As illustrated in Figure 7, students reported varying levels of happines when engaging in violent actions within games. A majority of 76\% reported feeling happy 
(46\% always, 30\% sometimes) when perfoming violent acts in games. In contrast only 24\% reported never enjoying violent acts in game or getting joy out of them.
This indicates that for most adolecents, engaging in violent gameplay is assosiated with positive emotional experiences. A notable difference can be oberved between
male and female adolescents. As for male students, as much as 62\% rewarted always enjoying violent in-geme acts and another 23\% reported sometimes enjoing them.
In contrast only 42\% of female adolescents reported always enjoying violent acts in game with another 34\% reporting sometimes enjoying them. This suggests that male
adolescents derive greater happines from violent gameplay compared to their female counterparts and are as such more emotionally engaged by violent in-game actions.\\

\subsection{Behavioral Outcomes Related to In-Game Violence}

\subsubsection{Aggression Level after prolonged Gaming}

\begin{wrapfigure}{r}{0.45\textwidth}
    \centering
    \includegraphics[width=0.45\textwidth]{images/Aggression growing after playing for a longer period.png}
    \caption{Self-Reported Aggression Levels After Prolonged Gaming Among Students}
    \label{fig: Aggression Levels}
\end{wrapfigure}
