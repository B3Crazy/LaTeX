\section{Introduction}
Video games have become one of the most prevalent forms of entertainment among adolescents across the globe,
with violent and death-related content present in many popular titles (\textit{e.g. Call of Duty, Fortnite, Grand Theft Auto}).
Existing rating systems such as PEGI (EU) and ESRB (US) (\cite{Tompson_Haninger_2004}) already flag the degree of violent realism in these games,
reflecting widespread public concern about the effects of such content on young people's development and moral reasoning.

\subsection{Background and significance}
Discussions over whether violent video games influence aggressive attitudes or behavior have continued since these games first appeared. 
Meta-analyses and longitudinal studies have produced mixed results, with some suggesting that frequent exposure to violent
or dark-themed games may increase tolerance toward aggression and reduce empathic concern for others 
(\cite{Anderson_Bushman_2018,Bushman_2025}). Other large-scale and well controlled studies, however, found little to no direct
causal link between exposure to violent game content and real-world aggression once individual traits and prior aggression are
accounted for (\cite{Przybylski_Warstein_2019,Lacko_Macháčková_Šmahel_2024}). \\
Although large-scale violent events such as school shootings and public attacks reopen public debates about the role of violent
media, including video games, in fostering aggressive tendencies, empirical evidence connecting real-world violent incidents and
video game consumption remains correlational and far from conclusive (\cite{Ramasubramanian_Banjo_2020}). Still, the perception
that violent video games might normalize aggression or death sustains an important social as well as scientific question:
\textbf{how does repeated exposure to virtual violence shape adolescents' moral reasoning, emotional responses to and perception
of real-world violence and harm and how different are these affects within the same demographic group?}

\subsection{Psychological mechanisms}
From a psychological perspective, video games can engage several mechanisms, such as reward-processing systems, that reinforce and
justify aggressive acts. In many games players are rewarded with (Experience-) points, level-ups, in-game items or currency after
committing violent or unjustful acts, such as slaying enemies or stealing from non-player characters (NPCs). This rewarding
may strengthen the player's association of a positive outcome with aggressive behavior, potentially leading to desensitization toward
violence over time (\cite{Carnagey_Anderson_Bushman_2007}). Additionally, the immersive and interactive nature of video games allows
players to actively participate in violent scenarios, which may enhance the emotional impact and identification with aggressive
characters. Furthermore, these games often lack realistic consequences for violent actions, which could distort players' understanding of the severity
and impact of violence in real life. And finally, in these games death is often portrayed in a trivialized or gamified manner,
where characters can respawn or continue playing after being killed, potentially diminishing the perceived gravity of death and harm (\cite{Hartmann_Krakowiak_Tsay-Vogel_2014}).

\subsection{Gender differences}
Gender differences may also play a role in how violent video games affect adolescents. Research indicates that boys statistically
play more violence and competitive-/combat-driven games than girls, who tend to prefer narrative-driven and cooperative experiences
(\cite{Walkerdine_2007,Hartmann_Klimmt_2017}). This differential exposure could lead to differences in moral reasoning and emotional
responses to violence if viewed as normative behavior within their gaming communities. Furthermore, societal norms and expectations
around masculinity and aggression may further shape these experiences and responses.

\subsection{Societal and developmental relevance}
Investigating the impact of adolescents' perception of violence and death in video games is especially relevant given that adolescents
are in a critical developmental stage where moral reasoning and empathy are still maturing (\cite{Blakemore_Robbins_2012}),
and they are particularly susceptible to peer influence and media effects (\cite{Steinberg_2008,Fikkers_Piotrowski_Lugtig_Valkenburg_2016}).
Adolescents are also in the process of forming their identity and understanding social norms, making them especially vulnerable and
receptive to media messages, including those related to violence and death (\cite{Avci_Baams_Kretschmer_2024}). Furthermore,
adolescents' digital lives often blur the boundaries between fantasy and reality, especially in immersive or multiplayer environments. \\
The issue is not only theoretical but also practical. Given the rise of interest in game and app development, understanding how
reward systems and death depictions shape users' psychological responses can lead the creation of games that are engaging, yet
ethically and emotionally responsible. Integrating psychological insights into game design could help developers build interactive
experiences that foster empathy, prosocial behavior, and critical thinking about violence rather than desensitization or normalization
of harm.

\subsection{Research question and hypotheses}
This study aims to investigate how exposure to violent and death-related content in video games influences adolescents'
moral reasoning, emotional responses, and behavioral tendencies. Specifically, the research will explore whether frequent exposure
to violent/death content is associated with increased tolerance for aggression, reduced empathy, and altered moral judgments concerning
harm to others. \\
Based on desensitization and aggression models, we hypothesize that:
\begin{enumerate}
    \item Greater exposure to violent/death-related game content correlates with higher acceptance of aggression and reduce empathy for
    pain and suffering in real life.
    \item Games that reward aggressive actions will amplify these associations compared to games without such reward systems.
    \item These relationships will differ by gender, with stronger effects expected among boys.
\end{enumerate}
Addressing these questions can clarify whether interactive violent content merely reflects existing preferences or actively shapes
and contributes to perceptual, moral and emotional desensitization. Understanding these dynamics is crucial for informing parents,
educators, policymakers and game developers about potential risks and ethical considerations in adolescent media consumption.