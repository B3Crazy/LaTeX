\section{Methodology}
\subsection{Research Design}
This study will employ a mixed-methods approach, combining a quantitative, cross-sectional correlational design with exploratory moderation analyses and examination
of potential relevant existing datasets. The primary data collection will involve administering standardized questionnaires with the aim to assess whether adolescents'
exposure to violent and death-related content can be associated with changing perceptions of violence, moral reasoning, emotional responses and behavioral tendencies.

\subsection{Participants}
The target for this study is adolescents aged 12 to 20 years. A sample size of approximately 100-200 participants will be recruited from secondary schools and clubs
to ensure a sufficient and diverse representation. The geo-locational focus will be the set on urban and suburban regions within Nordrhine-Westphalia, Germany. \\
For inclusion in the study, participants must access to video games at least twice a week. To ensure ethical compliance, informed consent will be obtained from all participants
and participation will be voluntary, anonymous and in accordance with ethical guidelines. For participants under the age of 14, parental consent will also be required.

\subsection{Materials and Measures}
A structured questionaire will be developed to assess the key constructs of interest. The questionnaire will include validated scales and self-report items to measure:
\begin{itemize}
    \item \textbf{Exposure to video games:} Frequency and duration of video games played, with a focus on violent and death-related content.
    \subitem e.g., "How often do you play video games?", "Do you play games that involve violence or death?"
    \item \textbf{Rewarding and supporting of aggressive acts:} Assessment of the extent to which the games played reward and support aggressive behavior.
    \subitem e.g., "In these games, how often does the gameplay require you to perform aggressive actions to succeed?", "In these games, how/ how often are aggressive
    actions rewarded?"
    \item \textbf{Moral reasoning about violence:} Using established scales such as the Aggression Acceptance Scale (AAS, \cite{Anderson_Bushman_2018}) to measure attitudes
    toward aggression.
    \subitem e.g., "Is it acceptable to hit someone who insults you?", "Is it okay to use violence to solve problems?"
    \item \textbf{Perception of death and empathy for pain:} Utilizing instruments like the Basic Empathy Scale (BES) or the Empathy for Pain Questionnaire
    (\cite{Miedzobrodzka_2023}) to assess emotional and cognitive empathy toward others' suffering.
    \subitem e.g., "When I see someone get hurt, I feel sorry for them.", "I can understand how others feel when they are in pain."
    \item \textbf{Demographics and Gender:} Collecting information on the participants age, gender and demographic background to explore potential moderating effects.
    \item \textbf{Control variables:} Including measures of average weekly playtime, game genre preference (competitive, narrative, cooperative), parental mediation
    (restrictive / active), and prior aggression incidents.
\end{itemize}

All questionnaires will be administered in English and will have a German translation available to ensure comprehension amongst all participants.

\subsection{Procedure}
The Procedure follows the same structure for all participants. Access will be made available with a link to the online questionnaire.
\begin{enumerate}
    \item \textbf{Distribution:} The distribution will be handled through multiple channels to reach a diverse and representative sample of adolescents.
        \subitem Schools: A person of authority will br contacted to distribute the link to students.
        \subitem Clubs: Trainers or leaders will be contacted to distribute the link to members.
    \item \textbf{Consent:} Prior to participation the questionnaire will inform the adolescents that participation is entirely voluntary. It will inform that children under
    the age of 14 have to take the survey with parental consent. Anonymity and confidentiality will be assured.
    \item \textbf{Questionnaire sequence:}
        \subitem Demographics, age and gender
        \subitem Video game exposure and habits
        \subitem Perception of violence, empathy and moral disengagement
    \item \textbf{Debriefing:} After finishing the questionnaires participants will receive an explanation on the purpose of the survey.
\end{enumerate}